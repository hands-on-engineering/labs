In the 1990s, Extensible Markup Language (XML) was developed by the W3C as a file format intended to structure data documents such that they were both easily readable by humans and also parsable in computer programs. Please note that XML is not the same thing as HTML; HTML is used to describe how a page should be presented in browsers, while XML describes content.


Later, Javascript Object Notation (JSON) and YAML were pushed as light-weight alternatives to XML. JSON is often used to send data back and forth between web clients and servers, while YAML has become popular for configuration files. Since YAML 1.2, YAML is fully compatible with JSON as an official subset. \cite{ben-kiki_evans_net_2009}. 


\lstinputlisting[language=XML, caption=Sample XML document \cite{xml_tutorial}]{code/sample-xml.xml}
\lstinputlisting[language=yaml, caption=Sample YAML document \cite{yaml_json_comparison}]{code/sample-yaml.yaml}
\lstinputlisting[language=json, caption=Sample JSON document \cite{yaml_json_comparison}]{code/sample-yaml.yaml}
